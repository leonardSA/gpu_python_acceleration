\section{Conclusion}

\texttt{PyOpenCL} est un \textit{framework} simple d'utilisation et de prise en main 
avec une documentation officielle explicite. À l'opposé,
\texttt{OpenCL} est un language complexe. Beaucoup de notions tels que les 
modèles d'exécution et de mémoire sont à comprendre.
Même l'installation peut poser un challenge, il n'y a pas (pour \texttt{Intel}) 
de guide d'installation officiel expliquant l'ensemble des étapes. On n'a malheureusement 
pas réussi à ``le vendre'' car aucun de nos résultats n'ont été concluants. 
Cependant, il y a une raison pour laquelle \texttt{OpenCL} existe et que les
accélarations sur d'autres composants que le CPU est en essort. En imaginant qu'on 
arrive à obtenir des résultats concluants, il serait possible d'appliquer celà 
aux calculs complexes existants dans les réseaux de neurones par exemple afin 
de pouvoir accélérer l'apprentissage de l'intelligence artificielle. Dans tous 
les cas, il serait pratique pour envisager l'optimisation de \textit{bottlenecks}.
Cependant, il faut y réfléchir conséquemment avant car cela demande beaucoup 
d'investissement.
