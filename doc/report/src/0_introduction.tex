\section{Introduction}

Ce document présente l'emploie d'\texttt{OpenCL} et du \textit{framework} 
\texttt{PyOpenCL}. Son but est de montrer comment il est possible de cibler
le GPU \textit{(Graphics Processing Unit)} pour obtenir une exécution parallèle 
avec Python dans le but d'optimiser certaines parties du code.

``\texttt{OpenCL} \textit{(Open Computing Language)} est la combinaison d’une API et d’un
langage de programmation dérivé du \texttt{C}, proposé comme un standard ouvert
par le \texttt{Khronos Group}.''\autocite{wikiopencl}. Il permet d'implémenter des 
algorithmes sur les GPU et est donc conçu pour faire du code parallèle.

Quant à \texttt{PyOpenCL}, 
c'est un \textit{framework} qui permet d'intégrer du code \textit{OpenCL} à
du \texttt{Python}. C'est un projet similaire à \texttt{PyCUDA} mais il a la 
particularité de ne pas être limité aux GPU \texttt{Nvidia}.
C'est aussi un projet 
\href{https://github.com/inducer/pyopencl}{\textit{open source}}.

On présentera donc d'abord les différentes architectures GPU existantes et \texttt{OpenCL}.
Puis on verra l'utilisation de \texttt{PyOpenCL}. Enfin, on montrera un exemple de développement 
de code \texttt{Python} et \texttt{OpenCL} pour la multiplication de matrices.

L'ensemble du code est disponible au lien suivant: \url{https://github.com/leonardSA/gpu\_python\_acceleration/}.
