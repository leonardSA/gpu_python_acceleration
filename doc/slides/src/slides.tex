\documentclass[9pt]{beamer}

\title{Accélération Python avec GPU}
\author{Joseph Leonard Stephen Auguste}
\usetheme{Goettingen}

\usepackage[utf8]{inputenc}
\usepackage[french]{babel}
\usepackage[T1]{fontenc}
\usepackage{hyperref}
\usepackage{listings}
\usepackage{graphicx}
\usepackage{dirtytalk}
\usepackage{xcolor}
\usepackage{lstautogobble}

\addtobeamertemplate{navigation symbols}{}{%
    \usebeamerfont{footline}%
    \usebeamercolor[fg]{footline}%
    \hspace{1em}%
    \insertframenumber/\inserttotalframenumber{}
}

\definecolor{codegreen}{rgb}{0,0.6,0}
\definecolor{codegray}{rgb}{0.5,0.5,0.5}
\definecolor{codepurple}{rgb}{0.58,0,0.82}
\definecolor{backcolour}{rgb}{0.95,0.95,0.92}

\lstdefinestyle{mystyle}{
    backgroundcolor=\color{backcolour},   
    commentstyle=\color{codegreen},
    keywordstyle=\color{magenta},
    numberstyle=\tiny\color{codegray},
    stringstyle=\color{codepurple},
    basicstyle=\ttfamily\footnotesize,
    breakatwhitespace=false,         
    breaklines=true,                 
    captionpos=b,                    
    keepspaces=true,                 
    numbers=left,                    
    numbersep=5pt,                  
    showspaces=false,                
    showstringspaces=false,
    showtabs=false,                  
    tabsize=2,
    belowskip=-0.5em,
    autogobble
}

\lstset{style=mystyle}


\begin{document}

\maketitle

\section{Introduction}
\begin{frame}
    \frametitle{Qu'est-ce qu'OpenCL?}
    \say{OpenCL (Open Computing Language) est la combinaison d'une API 
    et d'un langage de programmation dérivé du C, 
    proposé comme un standard ouvert par le Khronos Group.}
    \newline
    \-- Wikipédia\pause{}
    \vspace{20pt}
    \newline
    OpenCL permet:\pause{}
    \begin{itemize}
        \item Programmer sur GPU\pause{}
        \item Parallélisation\pause{}:
            \begin{itemize}
                \item Parallélisation de tâches (task parallelism)
                \item Parallélisation de données (data parallelism)
            \end{itemize}
    \end{itemize}\pause{}
    \vspace{20pt}
    Pour l'intégrer à Python on utilise le framework \textbf{PyOpenCL}.
\end{frame}

\section{Installations}
\begin{frame}
    \frametitle{Installation d'OpenCL}
    Deux éléments sont nécessaires au fonctionnement d'OpenCL:\pause{}
    \begin{itemize}
        \item Les headers: c'est l'API qui est définie par Khronos Group\pause{}
        \item Les runtimes: c'est l'implémentation qui est définie par le 
            vendeur du GPU (Nvidia, AMD ou Intel)
    \end{itemize}
\end{frame}

\begin{frame}[fragile]
    \frametitle{Installation des headers sous Linux}
    Une ligne de commande:
    \begin{lstlisting}
       sudo apt install opencl-headers 
    \end{lstlisting}
\end{frame}


\begin{frame}[fragile]
    \frametitle{Installation des runtimes Intel sous Linux}
    Installation des \textit{compute runtime} de Intel (architecture AMD64):
    \begin{lstlisting}
    mkdir neo
    cd neo
    wget https://github.com/intel/compute-runtime/releases/download/19.14.12751/intel-gmmlib_19.1.1_amd64.deb
    wget https://github.com/intel/compute-runtime/releases/download/19.14.12751/intel-igc-core_19.11.1622_amd64.deb
    wget https://github.com/intel/compute-runtime/releases/download/19.14.12751/intel-igc-opencl_19.11.1622_amd64.deb
    wget https://github.com/intel/compute-runtime/releases/download/19.14.12751/intel-opencl_19.14.12751_amd64.deb
    wget https://github.com/intel/compute-runtime/releases/download/19.14.12751/intel-ocloc_19.14.12751_amd64.deb
    sudo apt install ./*deb
    \end{lstlisting}
    \vspace{20pt}
    \textbf{Attention:} ne pas copier coller le texte, aller chercher les 
    paquets \href{https://github.com/intel/compute-runtime/releases}{ici}.
\end{frame}

\begin{frame}[fragile]
    \frametitle{Installation des runtimes Intel sous Linux}
    L'archive à récupérer se trouve 
    \href{https://software.intel.com/content/www/us/en/develop/articles/opencl-drivers.html}{ici} 
    dans la section \say{Intel CPU Runtime for OpenCL}:
    \begin{lstlisting}[language=sh]
    tar -xvf l_opencl_p_18.1.0.015.tgz
    cd l_opencl_p_18.1.0.015/rpm
    # requires alien and libnuma1 - converts everything to deb packages
    alien *.rpm
    sudo apt install ./*deb
    # makes directories for vendors
    sudo mkdir -p /usr/lib/OpenCL/vendors/
    sudo mv /opt/intel /usr/lib/OpenCL/vendors/
    sudo cp /usr/lib/x86_64-linux-gnu/libOpenCL.so /usr/lib/OpenCL/vendors/intel/libOpenCL.so
    # configure dynamic linking
    sudo echo "/usr/lib/OpenCL/vendors/intel" > /etc/ld.so.conf.d/opencl-vendor-intel.conf
    sudo ldconfig
    \end{lstlisting}
    \vspace{20pt}
    \textbf{Attention:} les chemins ne seront pas forcément les mêmes.
\end{frame}

\begin{frame}[fragile]
    \frametitle{Installation alternative des runtimes Intel sous Linux}
    Il est possible d'installer les runtimes en une ligne de commande:
    \begin{lstlisting}[language=sh]
    apt-get install beignet beignet-dev
    \end{lstlisting}
    \vspace{20pt}
    \textbf{Attention:} Beignet n'est plus recommandé/maintenu par Intel 
    (\href{https://software.intel.com/en-us/forums/opencl/topic/758168}{source}).
    \newline
    \textbf{Non testé par moi.}
\end{frame}

\begin{frame}[fragile]
    \frametitle{Installation de PyOpenCL}
    Prérequis:
    \begin{itemize}
        \item Python installé (de préférence Python3)
        \item Pip installé (de préférence Pip3)
    \end{itemize}
    \vspace{20pt}
    Installation en une ligne de commande:
    \begin{lstlisting}[language=sh]
        pip3 install pyopencl
    \end{lstlisting}
\end{frame}

\end{document}
