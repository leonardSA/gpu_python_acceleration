\documentclass[8pt]{beamer}

\title{Accélération Python avec GPU}
\author{Joseph Leonard Stephen Auguste}
\usetheme{Goettingen}

\usepackage[utf8]{inputenc}
\usepackage[french]{babel}
\usepackage[T1]{fontenc}
\usepackage[hidelinks]{hyperref}
\usepackage{listings}
\usepackage{amsmath}
\usepackage{graphicx}
\usepackage{float}
\usepackage{dirtytalk}
\usepackage{xcolor}
\usepackage[toc,page]{appendix}
\usepackage{lstautogobble}
\usepackage[left=2cm, right=2cm, top=2cm, bottom=2cm]{geometry}
\usepackage{csquotes}
\usepackage[backend=bibtex]{biblatex}

\bibliography{references}

\renewcommand{\appendixtocname}{Annexes}
\renewcommand{\appendixpagename}{Annexes}

\definecolor{codegreen}{rgb}{0,0.6,0}
\definecolor{codegray}{rgb}{0.5,0.5,0.5}
\definecolor{codepurple}{rgb}{0.58,0,0.82}
\definecolor{backcolour}{rgb}{0.95,0.95,0.92}

\lstdefinestyle{mystyle}{
    backgroundcolor=\color{backcolour},   
    commentstyle=\color{codegreen},
    keywordstyle=\color{magenta},
    numberstyle=\tiny\color{codegray},
    stringstyle=\color{codepurple},
    basicstyle=\ttfamily\footnotesize,
    breakatwhitespace=false,         
    breaklines=true,                 
    captionpos=b,                    
    keepspaces=true,                 
    numbers=left,                    
    numbersep=5pt,                  
    showspaces=false,                
    showstringspaces=false,
    showtabs=false,                  
    tabsize=2,
    belowskip=-0.5em,
    autogobble
}

\lstset{style=mystyle}


\begin{document}

\maketitle

\section{Introduction}
\begin{frame}
    \frametitle{Qu'est-ce qu'OpenCL?}
    \say{OpenCL (Open Computing Language) est la combinaison d'une API 
    et d'un langage de programmation dérivé du C, 
    proposé comme un standard ouvert par le Khronos Group.}
    \newline
    \-- Wikipédia\pause{}
    \vspace{20pt}
    \newline
    OpenCL permet:\pause{}
    \begin{itemize}
        \item Programmer sur GPU\pause{}
        \item Parallélisation\pause{}:
            \begin{itemize}
                \item Parallélisation de tâches (task parallelism): utile si on 
                    cherche à exécuter des programmes différents en parallèle\pause{}
                \item Parallélisation de données (data parallelism): utile si on 
                    cherche à modifier différentes parties des données avec une 
                    même opération
            \end{itemize}
    \end{itemize}\pause{}
    \vspace{20pt}
    Pour l'intégrer à Python on utilise le framework \textbf{PyOpenCL}.
\end{frame}

\section{Installations}
\begin{frame}
    \frametitle{Installation d'OpenCL}
    Deux éléments sont nécessaires au fonctionnement d'OpenCL:\pause{}
    \begin{itemize}
        \item Les headers: c'est l'API qui est définie par Khronos Group\pause{}
        \item Les runtimes: c'est l'implémentation qui est définie par le 
            vendeur du GPU (Nvidia, AMD ou Intel)
    \end{itemize}
\end{frame}

\begin{frame}[fragile]
    \frametitle{Installation des headers sous Linux}
    Une ligne de commande:
    \begin{lstlisting}
       sudo apt install opencl-headers 
    \end{lstlisting}
\end{frame}


\begin{frame}[fragile]
    \frametitle{Installation des runtimes Intel sous Linux}
    Installation des \textit{compute runtime} de Intel (architecture AMD64):
    \begin{lstlisting}
    mkdir neo
    cd neo
    wget https://github.com/intel/compute-runtime/releases/download/19.14.12751/intel-gmmlib_19.1.1_amd64.deb
    wget https://github.com/intel/compute-runtime/releases/download/19.14.12751/intel-igc-core_19.11.1622_amd64.deb
    wget https://github.com/intel/compute-runtime/releases/download/19.14.12751/intel-igc-opencl_19.11.1622_amd64.deb
    wget https://github.com/intel/compute-runtime/releases/download/19.14.12751/intel-opencl_19.14.12751_amd64.deb
    wget https://github.com/intel/compute-runtime/releases/download/19.14.12751/intel-ocloc_19.14.12751_amd64.deb
    sudo apt install ./*deb
    \end{lstlisting}
    \vspace{20pt}
    \textbf{Attention:} ne pas copier coller le texte, aller chercher les 
    paquets \href{https://github.com/intel/compute-runtime/releases}{ici}.
\end{frame}

\begin{frame}[fragile]
    \frametitle{Installation des runtimes Intel sous Linux}
    L'archive à récupérer se trouve 
    \href{https://software.intel.com/content/www/us/en/develop/articles/opencl-drivers.html}{ici} 
    dans la section \say{Intel CPU Runtime for OpenCL}:
    \begin{lstlisting}[language=sh]
    tar -xvf l_opencl_p_18.1.0.015.tgz
    cd l_opencl_p_18.1.0.015/rpm
    # requires alien and libnuma1 - converts everything to deb packages
    alien *.rpm
    sudo apt install ./*deb
    # makes directories for vendors
    sudo mkdir -p /usr/lib/OpenCL/vendors/
    sudo mv /opt/intel /usr/lib/OpenCL/vendors/
    sudo cp /usr/lib/x86_64-linux-gnu/libOpenCL.so /usr/lib/OpenCL/vendors/intel/libOpenCL.so
    # configure dynamic linking
    sudo echo "/usr/lib/OpenCL/vendors/intel" > /etc/ld.so.conf.d/opencl-vendor-intel.conf
    sudo ldconfig
    \end{lstlisting}
    \vspace{20pt}
    \textbf{Attention:} les chemins ne seront pas forcément les mêmes.
\end{frame}

\begin{frame}[fragile]
    \frametitle{Installation alternative des runtimes Intel sous Linux}
    Il est possible d'installer les runtimes en une ligne de commande:
    \begin{lstlisting}[language=sh]
    apt-get install beignet beignet-dev
    \end{lstlisting}
    \vspace{20pt}
    \textbf{Attention:} Beignet n'est plus recommandé/maintenu par Intel 
    (\href{https://software.intel.com/en-us/forums/opencl/topic/758168}{source}).
    \newline
    \textbf{Non testé par moi.}
\end{frame}

\begin{frame}[fragile]
    \frametitle{Installation de PyOpenCL}
    Prérequis:
    \begin{itemize}
        \item Python installé (de préférence Python3)
        \item Pip installé (de préférence Pip3)
    \end{itemize}
    \vspace{20pt}
    Installation en une ligne de commande:
    \begin{lstlisting}[language=sh]
        pip3 install pyopencl
    \end{lstlisting}
\end{frame}

\section{OpenCL avec PyOpenCL}
\begin{frame}[fragile]
    \frametitle{Créer le contexte d'exécution}
    Pour pouvoir exécuter des programmes OpenCL dit \textbf{Kernel} 
    il faut d'abord choisir son support, créer son contexte d'exécution, 
    le compiler (build) et enfin ajouter ce contexte à la file de commandes.
    \begin{lstlisting}[language=Python]
    import pyopencl as cl
    # choose first platform 
    platform = cl.get_platforms()[0]       
    # retrive platform devices to create context
    devices = platform.get_devices()        
    # create context for platform
    context = cl.Context(devices=devices)    
    # add OpenCL source code to context (source is a string)
    program_source = cl.Program(context, source)
    # compile the kernel
    program = program_source.build()
    # enqueue the context => make the builded programs avaiable for execution
    queue = cl.CommandQueue(context)
    \end{lstlisting}
\end{frame}

\section{OpenCL avec PyOpenCL}
\begin{frame}[fragile]
    \frametitle{Création et copies de buffers}
    Afin de pouvoir transférer des données au programme OpenCL il faut 
    commencer par définir des buffers. 
    \newline
    \textbf{Attention:} le type est très important (d'où l'utilisation de la 
    librarie numpy), il faut se référer à la documentation officielle.
    \begin{lstlisting}[language=Python]
    import numpy as np
    import pyopencl as cl
    # instantiate a numpy.ndarray of N elements of type float32 with random values
    # floats are encoded in 32bits in OpenCL
    array_in = np.random.rand(N).astype(np.float32)
    # instantiate a similar numpy.ndarray but filled with zeros
    array_out = np.zeros(N, dtype=float32)
    # context is the one created in the previous slide
    # create a read only buffer using array_in
    buffer_in = cl.Buffer(context, flags=cl.mem_flags.READ_ONLY,
                          size=array_in.nbytes)
    # create a write only buffer using array_out
    array_out = cl.Buffer(context, flags=cl.mem_flags.WRITE_ONLY,
                          size=b.nbytes)
    # copy array_in content onto GPU via buffer_in
    # is_blocking flag set to True by default
    cl.enqueue_copy(queue, src=array_in, dest=buffer_in)    
    # ... imagine execution which fills buffer_out (detailed further in the slides) ...
    # copy array_out content off GPU and onto host via buffer_out
    cl.enqueue_copy(queue, src=buffer_out, dest=array_out)
    \end{lstlisting}
\end{frame}

\section{OpenCL avec PyOpenCL}
\begin{frame}[fragile]
    \frametitle{Exécution d'un kernel}
    L'exécution d'un Kernel se fait en appelant l'appelant sur la variable 
    \textit{program} définie précedemment.
    Disons ici que nous avons un Kernel du nom de \textit{inc} qui incrémente 
    chaque élément d'un vecteur.
    \begin{lstlisting}[language=Python]
    # ...Buffer creations and copy onto GPU....
    # Kernel function prototype: kernel void inc(global float * in, global float * out)
    kernel_arguments = (buffer_in, buffer_out)
    # run the program
    program.inc(queue,                  # aforementioned queue
                len(array_in),          # global memory
                None,                   # local memory : None if not used
                *kernel_arguments)
    # ...Copy buffer off GPU...
    \end{lstlisting}
\end{frame}

\section{OpenCL avec PyOpenCL}
\begin{frame}[fragile]
    \frametitle{Définition du Kernel}
    Le Kernel est défini dans une chaîne de caractère 
    (peut aussi être lu d'un fichier).
    \begin{lstlisting}[language=python]
    program_source = """
        kernel void inc(global float * in, global float * out) {
            int id = get_global_id(0);
            out[id] = in[id] + 1;
        }
    """
    \end{lstlisting}
\end{frame}

\section{OpenCL avec PyOpenCL}
\begin{frame}
    \frametitle{Notes (non négligeables)}
    Les fonctions de PyOpenCL:\@ \textit{enqueue\_copy} et du genre \textit{program.inc}
    retournent un objet de la classe pyopencl.Event. On peut appeller certaines méthodes
    sur ces objets dont notamment \textit{wait()} qui permet d'attendre la fin de 
    l'exécution de l'événement.
    \vspace{20pt}
    \newline
    L'allocation dynamique ainsi que la récursivité \textbf{n'existent pas en OpenCL.}
\end{frame}

\end{document}
