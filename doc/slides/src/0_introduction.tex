\begin{frame}
    \frametitle{Les outils}
    \say{OpenCL (Open Computing Language) est la combinaison d'une API 
    et d'un langage de programmation dérivé du C, 
    proposé comme un standard ouvert par le Khronos Group.}
    \newline
    \-- Wikipédia\pause{}
    \vspace{20pt}
    \newline
    OpenCL permet:\pause{}
    \begin{itemize}
        \item Programmer sur GPU\pause{}
        \item Parallélisation:
            \begin{itemize}
                \item Parallélisation de tâches \textit{(task parallelism)}
                \item Parallélisation de données \textit{(data parallelism)}
            \end{itemize}
    \end{itemize}\pause{}
    \vspace{20pt}
    PyOpenCL:\pause{}
    \begin{itemize}
        \item \textit{Framework} pour intégrer du code OpenCL à du code Python\pause{}
        \item Projet \textit{open source}: \url{github.com/inducer/pyopencl}
    \end{itemize}\pause{}
    \vspace{20pt}
    Code présenté: \url{github.com/leonardSA/gpu_python_acceleration}
\end{frame}


